\documentclass[10pt,letterpaper]{article}

\usepackage{amsmath}
\usepackage{}
% \usepackage{mathbb}
% \usepackage[amsmath]{ntheorem}
\usepackage{amsfonts}
\usepackage{graphicx}
\usepackage{fullpage}
\usepackage{hyperref}
% \usepackage[export]{adjustbox}
\usepackage[caption=false]{subfig}

\DeclareMathOperator*{\argmax}{arg\,max}
\DeclareMathOperator*{\argmin}{arg\,min}
\newtheorem{definition}{Definition}
\newtheorem{example}{Example}

\newcommand{\funman}{FUNMAN}
\newcommand{\region}{\bf X}
\newcommand{\posregion}{{\region}^+}
\newcommand{\negregion}{{\region}^-}
\newcommand{\irrelevantregion}{{\region}^\oslash}
\newcommand{\point}{{\bf x}}
\newcommand{\model}{{\bf M}}
\newcommand{\query}{{\bf Q}}
\newcommand{\bounds}{{\bf B}}
\newcommand{\parameters}{{\bf p}}
\newcommand{\parameterbox}{{\bf p}^{[]}}
\newcommand{\reals}{{\mathbb{R}}}

% \setlength{\floatsep}{ 1.0pt plus 2.0pt minus 2.0pt}

\title{\funman Abstraction}
\author{Dan Bryce}
\begin{document}
\maketitle

\section{Introduction}

We describe methods to stratify and abstract (de-stratify) Petrinets for compartmental models.  The motivation for abstracting a Petrinet is to reduce its size, which becomes exponential in the number of stratified variables.  Reducing the size of the model can have a significant impact on runtime, and it is possible to answer useful queries with the abstract model.  The following sections include a background (defining the models), a description of abstraction, an approach needed to bound the abstracted models (or to perform parameter synthesis directly in a simulator), and then a comparison of simulation results for several variations of a baseline model that applies stratification, bounding, and abstraction.  

\section{Background}
\begin{definition}
    A Petrinet $\Omega$ is a directed graph $(V, E)$ with vertices $V=(V_x,
    V_z)$ partitioned into sets $V_x$ of state vertices and $V_z$ of transition
    vertices, and edges $E=(E_{in}, E_{out})$ partitioned into sets $E_{out}$ of
    flow-out and $E_{in}$ flow-in edges. 
\end{definition}

\begin{definition}
A flow-out edge $e \in E_{out}$ comprises a pair of vertices $(v_x,v_z)$, where
$v_x \in V_x$ is a state vertex, $v_z \in V_z$ is a transition vertex, and the
flow is directed from $v_x$ to $v_z$.  
\end{definition}

\begin{definition}
    A flow-in edge $e \in E_{in}$ comprises a pair of vertices $(v_z,v_x)$,
    similar to a flow-out edge, except that the flow is directed from $v_z$ to
    $v_x$.  
\end{definition}



\begin{definition}
    The ODE semantics $\Theta$ of the Petrinet $\Omega$ defines a tuple $(P, X,
    Z, {\cal I}, {\cal P}, {\cal X}, {\cal Z}, {\cal R})$ where 
    \begin{itemize}
        \item $P$ is a set of parameters;
        \item $X$ is a set of state variables;
        \item $Z$ is a set of transitions;
        \item ${\cal I}: S \rightarrow \reals$ assigns the initial value of
        state variables to a real number;
        \item ${\cal P}: P \rightarrow \reals \cup \reals \times \reals$ assigns
        parameters to a real number, or a pair of real numbers defining an
        interval;
        \item ${\cal X}: X \rightarrow V_x$ assigns state variables to state
        vertices;
        \item ${\cal Z}: Z \rightarrow V_z$ assigns transtions to transition
        vertices; and
        \item ${\cal R}: {\bf P} \times {\bf X} \times Z \rightarrow \reals$
        defines the rate of each transition in $x \in X$ in terms of the set of
        parameter vectors ${\bf P}$ and state variable vectors ${\bf X}$.  
    \end{itemize}
    The elements of the Petrinet $\Omega$ and semantics $\Theta$ define the
    partial derivative $\frac{d {\bf x}}{dt}$, so that for each state variable
    $x \in X$:
    
    \begin{equation}\label{eqn:flow}
        \frac{dx}{dt} = \sum_{v_z \in V_z^{in(x)}} {\cal R}({\bf p}, {\bf x}, z) - \sum_{v_z \in V_z^{out(x)} } {\cal R}({\bf p}, {\bf x}, z)
    \end{equation}
\noindent where $V_z^{in(x)} = \{v_z \in V_z | (v_z, v_x) \in E_{in}\}$ and
    $V_z^{out(x)}=\{v_z \in V_z| (v_x, v_z) \in E_{out}\}$ are the transition
    vertices that flow in and out of the vertex $v_x$, respectively. We denote
    by $\nabla_{\Omega, \Theta}({\bf p}, {\bf x}, t) = (\frac{dx_1}{dt},
    \frac{dx_2}{dt}, \ldots)^T$, the gradient comprised of components in
    Equation \eqref{eqn:flow}.
\end{definition}

Using the partial derivatives defined by the Petrinet graph and semantics, we
can define the state vector at given time $t+dt$ with the forward Euler method
as:

\begin{eqnarray*}
    \frac{d {\bf x}}{dt} &=& \nabla_{\Omega, \Theta}({\bf p},{\bf x}, t)\\
    \frac{{\bf x}(t+dt)-{\bf x}(t)}{dt} &=& f_{\Omega, \Theta}({\bf p},{\bf x},
    t)\\
    {\bf x}(t+dt)&=& f_{\Omega, \Theta}({\bf p},{\bf x}, t)dt+ {\bf x}(t)
\end{eqnarray*}

\begin{definition}
    An abstraction $(\Theta', \Omega')$ of a Petrinet and the associated
    semantics $(\Theta, \Omega)$ that is produced by the abstraction operator
    $A$ has the following properties:
    \begin{itemize}
        \item State: For each $x \in X$,  $A(x) = x'$, where $x' \in
        X'$.  For each vertex $v_x \in V_x$,  $A(v_x) = v_x'$ where $v_x' \in
        V_x'$.   For each $x\in X$ where  ${\cal X}(x) =
        V_x$, $A(x) = x'$, and $A(v_x) = v_x'$, then ${\cal X}'(x')=
        v_{x'}'$.  For each $x' \in X'$, ${\cal X}'(x') = \sum\limits_{x \in X: A(x) = x'} {\cal X}(x)$.
        \item Parameters: For each $p \in P$, $A(p) = p'$, where $p'\in P'$.
        For each $p' \in P'$, ${\cal P}'(p') = \sum\limits_{p \in P: A(p) = p'} {\cal P}(p)$.
        \item Transitions: For each $z \in Z$, $A(z) = z'$, where $z' \in Z'$.
        For each vertex $v_z \in V_z$, $A(v_z) = v_z'$, where $v_z' \in V_z'$.
        For each $z \in Z$, where ${\cal
        Z}(z) = v_z$, $A(z) = z'$, and $A(v_z) = v_z'$, then ${\cal
        Z}'(z') = v_{z'}'$. 
        \item In Edges: For each edge $(v_z, v_x) \in E_{in}$, $A((v_z, v_x)) =
        (v_z', v_x')$, $A(v_x) = v_x'$, and $A(v_z) = v_z'$, where $(v_z',
        v_x')\in E_{in}'$;
        \item Out Edges: For each edge $(v_x, v_z) \in E_{out}$, $A((v_x, v_z))
        = (v_x', v_z')$; $A(v_x) = v_x'$, and $A(v_z) = v_z'$, where $(v_x',
        v_z')\in E_{out}'$;

        
        \item Transition Rates: For each $z' \in Z'$, ${\cal R}'({\bf p}', {\bf
        x}', z') = \sum\limits_{z \in Z: A(z)=z'} {\cal R}({\bf p}, {\bf
        x}, z)$.
    \end{itemize}
\end{definition}

The abstraction $(\Theta', \Omega')$ similarly defines the gradient $\nabla_{\Omega', \Theta'}({\bf p}', {\bf x}', t) = (\frac{dx_1'}{dt},
\frac{dx_2'}{dt}, \ldots)^T$, in terms of Equation \ref{eqn:flow}.
The abstraction thus expresses the gradient by aggregating terms from the
base Petrinet and semantics.  It preserves the flow on transitions, but
expresses the transition rates in terms of the base states.  As such, the
abstraction compresses the Petrinet graph structure, but at the cost of
expanding the expressions for transition rates. Moreover, the transition
rates refer to state variables and parameters that are not expressed
directly by the Petrinet and semantics, and by extension, the gradient. 

We modify the abstraction in what we call a \emph{bounded abstraction}, so that
it refers to the abstract, and not the base, Petrinet and semantics.  This
bounded abstraction replaces base elements with corresponding bounded elements.
For example, if $A(x_1) = x'$ and $A(x_2) = x'$ ($x_1$ and $x_2$ are base
variables represented by $x'$ in the abstraction), a possible transition rate
could be of the form
${\cal R}'({\bf p}', {\bf x}', z') = p_1 x_1 + p_2 x_2$.  By construction, we
know that $x_1 + x_2 = x'$.  However, in general $p_1 \not= p_2$, and we cannot
say that $p_1 x_1 + p_2 x_2 = p'x'$ for some $p'$.  Yet, if we replace $p_1$ and
$p_2$ by $p^{ub} = \max(p_1, p_2)$, then $p^{ub} x_1 + p^{ub} x_2 \geq p'x'$.  Simplifying, we
get $p^{ub} x_1 + p^{ub} x_2 = p^{ub}(x_1 + x_2) = p^{ub} x' \geq p'x'$.  A
similar argument can be made where $p^{lb} = \min(p_1, p_2)$ and we find that
$p^{lb} x' \leq p'x'$.  

By introducing the bounded parameters, we no longer
rely upon the base state variables or parameters.  However, in tracking the
effect of the bounded
parameters, the bounded abstraction must also track bounded rates and bounded
state variables.  The resulting bounded abstraction thus over-approximates the
abstraction and base model, wherein we can derive bounds on the state variables
at each time, which may correspond to a larger (hence over-approximation) set of
state trajectories.

\begin{definition}
A bounded abstraction $(\Theta^B, \Omega^B)$ of an abstraction $(\Theta',
\Omega')$ of $(\Theta, \Omega)$ replaces each element of $(\Theta',
\Omega')$ by a pair of elements denoting the lower and upper bound of that
element (and referred to with the ``$lb$'' and ``$ub$'' superscripts).  The
bounded abstraction defines:
\begin{itemize}
    \item State: For each $x' \in X'$,  $x^{lb}, x^{ub} \in X^B$.  For each
    $v_{x'}' \in V_x'$, ${\cal X}^B(x^{lb}) = v_{x^{lb}}^B$ and ${\cal X}^B(x^{ub}) =
    v_x^{ub}$.   For each $x^{lb}, x^{ub} \in X^B$, ${\cal I}^B(x^{lb}) = {\cal
    I}^B(x^{ub}) = {\cal I}'(x')$.
        \item Parameters: For each $p' \in P'$,
        let ${\cal P}^B(p^{lb}) = \min\limits_{p \in P: A(p) = p'} {\cal P}(p)$ and ${\cal P}^B(p^{ub}) = \max\limits_{p \in P: A(p) = p'} {\cal P}(p)$. 
        

        \item Transitions: For each $z' \in Z'$, $z^{lb}, z^{ub} \in Z^B$.
        For each vertex $v_z \in V_z$, $v_{z^{lb}}, v_{z^{ub}} \in V_z^B$.
        
        \item In Edges: For each edge $(v_{z'}, v_{x'}) \in E_{in}'$, $(v_{z^{lb}},
        v_{x^{lb}}), (v_{z^{ub}},
        v_{x^{ub}}) \in E^B_{in}$.
        \item Out Edges: For each edge $(v_{x'}, v_{z'}) \in E_{out}'$, $(v_{x^{ub}},
        v_{z^{lb}}), (v_{x^{lb}},
        v_{z^{ub}}) \in E^B_{out}$.

        
        \item Transition Rates: For each $z^{lb} \in Z^B$, ${\cal R}^B({\bf p}^B, {\bf
        x}^B, z^{lb}) = \min\limits_{z \in Z: A(z)=z'} {\cal R}({\bf p}, {\bf
        x}, z)$ (replacing ${\bf p}$ and $[\bf x]$ of the minimal rate by the 
        elements in ${\bf p}^B$ and ${\bf x}^B$ respectively, which minimize the
        rate), and ${\cal R}^B({\bf p}^B,
        {\bf
        x}^B, z^{ub}) = \max\limits_{z \in Z: A(z)=z'} {\cal R}({\bf p}, {\bf
        x}, z)$ (similarly replacing ${\bf p}$ and ${\bf x}$ of the maximal rate by the 
        elements in ${\bf p}^B$ and ${\bf x}^B$ respectively, which maximize the
        rate).
\end{itemize}
    
\end{definition}

\section{Abstraction}
\begin{definition}
    An abstraction $(\Theta', \Omega')$ of a Petrinet and the associated
    semantics $(\Theta, \Omega)$ that is produced by the abstraction operator
    $A$ has the following properties:
    \begin{itemize}
        \item State: For each $x \in X$,  $A(x) = x'$, where $x' \in
        X'$.  For each vertex $v_x \in V_x$,  $A(v_x) = v_x'$ where $v_x' \in
        V_x'$.   For each $x\in X$ where  ${\cal X}(x) =
        V_x$, $A(x) = x'$, and $A(v_x) = v_x'$, then ${\cal X}'(x')=
        v_{x'}'$.  For each $x' \in X'$, ${\cal X}'(x') = \sum\limits_{x \in X: A(x) = x'} {\cal X}(x)$.
        \item Parameters: For each $p \in P$, $A(p) = p'$, where $p'\in P'$.
        For each $p' \in P'$, ${\cal P}'(p') = \sum\limits_{p \in P: A(p) = p'} {\cal P}(p)$.
        \item Transitions: For each $z \in Z$, $A(z) = z'$, where $z' \in Z'$.
        For each vertex $v_z \in V_z$, $A(v_z) = v_z'$, where $v_z' \in V_z'$.
        For each $z \in Z$, where ${\cal
        Z}(z) = v_z$, $A(z) = z'$, and $A(v_z) = v_z'$, then ${\cal
        Z}'(z') = v_{z'}'$. 
        \item In Edges: For each edge $(v_z, v_x) \in E_{in}$, $A((v_z, v_x)) =
        (v_z', v_x')$, $A(v_x) = v_x'$, and $A(v_z) = v_z'$, where $(v_z',
        v_x')\in E_{in}'$;
        \item Out Edges: For each edge $(v_x, v_z) \in E_{out}$, $A((v_x, v_z))
        = (v_x', v_z')$; $A(v_x) = v_x'$, and $A(v_z) = v_z'$, where $(v_x',
        v_z')\in E_{out}'$;

        
        \item Transition Rates: For each $z' \in Z'$, ${\cal R}'({\bf p}', {\bf
        x}', z') = \sum\limits_{z \in Z: A(z)=z'} {\cal R}({\bf p}, {\bf
        x}, z)$.
    \end{itemize}
\end{definition}

\begin{example}
    The abstraction $(\Theta', \Omega')$ of the stratified SIR model defines
    (with the changed elements highlighted by ``*''):
    \begin{eqnarray*}
        A &=& \left\{ 
            \begin{array}{lll}
                S &: S_1 &*\\
                S &: S_2&*\\
                I &: I\\
                R &: R\\
               \beta &: \beta_1&*\\
               \beta &: \beta_2&*\\
               \gamma &: \gamma\\
               inf&: inf_1&*\\
               inf&: inf_2&*\\
               rec&: rec\\
               v_S &: v_{S_1}&*\\
               v_S &: v_{S_2}&*\\
               v_I &: v_{I}\\
               v_R &: v_{R}\\
               (v_{S}, v_{inf}) &: (v_{S_1}, v_{inf_1})&*\\
               (v_{S}, v_{inf}) &: (v_{S_2}, v_{inf_2})&*\\
               (v_{I}, v_{inf}) &: (v_{I}, v_{inf_1})&*\\
               (v_{I}, v_{inf}) &: (v_{I}, v_{inf_2})&*\\
               (v_I, v_{rec}) &: (v_I, v_{rec})\\
               (v_{inf}, v_I) &: (v_{inf_1}, v_I)&*\\
               (v_{inf}, v_I) &: (v_{inf_2}, v_I)&*\\
               (v_{rec}, v_R) &: (v_{rec}, v_R)\\
            \end{array}\right.\\
            {\cal R} &=& \left\{ 
            \begin{array}{lll}
                \beta_1 S_1 I +  \beta_2 S_2 I& : z_{inf}&*\\
                \gamma I R & : z_{rec}\\
            \end{array}\right.\\
    \end{eqnarray*}
\end{example}

The abstraction $(\Theta', \Omega')$ similarly defines the gradient $\nabla_{\Omega', \Theta'}({\bf p}', {\bf x}', t) = (\frac{dx_1'}{dt},
\frac{dx_2'}{dt}, \ldots)^T$, in terms of Equation \ref{eqn:flow}.
The abstraction thus expresses the gradient by aggregating terms from the
base Petrinet and semantics.  It preserves the flow on transitions, but
expresses the transition rates in terms of the base states.  As such, the
abstraction compresses the Petrinet graph structure, but at the cost of
expanding the expressions for transition rates. Moreover, the transition
rates refer to state variables and parameters that are not expressed
directly by the Petrinet and semantics, and by extension, the gradient. 



\section{Bounding Petrinets}
We modify the abstraction in what we call a \emph{bounded abstraction}, so that
it refers to the abstract, and not the base, Petrinet and semantics.  This
bounded abstraction replaces base elements with corresponding bounded elements.
For example, if $A(S_1) = S$ and $A(S_2) = S$ ($S_1$ and $S_2$ are base
variables represented by $S$ in the abstraction), the transition rate associated with the $inf$ transition is
 ${\cal R}'({\bf p}', {\bf x}', z_{inf}) = \beta_1 S_1 I +  \beta_2 S_2 I$.
By construction, we know that $S_1 + S_2 = S$.  However, in general $\beta_1 \not=
\beta_2$, and we cannot say that $\beta_1 S_1 I + \beta_2 S_2 I = \beta S I$ for some definition of $\beta$.  Yet, if
we replace $\beta_1$ and $\beta_2$ by $\beta^{ub} = \max(\beta_1, \beta_2)$, then $\beta^{ub} S_1 I +
\beta^{ub} S_2 I \geq \beta S I$.  Simplifying, we get $\beta^{ub} S_1 I + \beta^{ub} S_2 I =
\beta^{ub}(S_1 + S_2)I = \beta^{ub} S I \geq \beta S I$.  A similar argument can be made for the lower bound where
$\beta^{lb} = \min(\beta_1, \beta_2)$ and we find that $\beta^{lb} S I \leq \beta S I$.  

By introducing the bounded parameters, we no longer rely upon the base state
variables or parameters.  However, in tracking the effect of the bounded
parameters, the bounded abstraction must also track bounded rates and bounded
state variables.  The resulting bounded abstraction thus over-approximates the
abstraction and base model, wherein we can derive bounds on the state variables
at each time, which may correspond to a larger (hence over-approximation) set of
state trajectories.

\begin{definition}
A bounded abstraction $(\Theta^B, \Omega^B)$ of an abstraction $(\Theta',
\Omega')$ of $(\Theta, \Omega)$ replaces each element of $(\Theta', \Omega')$ by
a pair of elements denoting the lower and upper bound of that element (and
referred to with the ``$lb$'' and ``$ub$'' superscripts).  The bounded
abstraction defines:
\begin{itemize}
    \item State: For each $x' \in X'$,  $x^{lb}, x^{ub} \in X^B$.  For each
    $v_{x'}' \in V_x'$, ${\cal X}^B(x^{lb}) = v_{x^{lb}}^B$ and ${\cal
    X}^B(x^{ub}) = v_{x^{ub}}^B$.   For each $x^{lb}, x^{ub} \in X^B$, ${\cal
    I}^B(x^{lb}) = {\cal I}^B(x^{ub}) = {\cal I}'(x')$.
        \item Parameters: For each $p' \in P'$, let ${\cal P}^B(p^{lb}) =
        \min\limits_{p \in P: A(p) = p'} {\cal P}(p)$ and ${\cal P}^B(p^{ub}) =
        \max\limits_{p \in P: A(p) = p'} {\cal P}(p)$. 
        

        \item Transitions: For each $z' \in Z'$, $z^{lb}, z^{ub} \in Z^B$. For
        each vertex $v_z \in V_z$, if $A(v_z)=v_z'$ then $v_{z^{lb}}^B, v_{z^{ub}}^B \in V_z^B$.
        
        \item In Edges: For each edge $(v_{z'}^B, v_{x'}^B) \in E_{in}'$,
        $(v_{z^{lb}}^B, v_{x^{lb}}^B), (v_{z^{ub}}^B, v_{x^{ub}}^B) \in E^B_{in}$.
        \item Out Edges: For each edge $(v_{x'}^B, v_{z'}^B) \in E_{out}'$,
        $(v_{x^{ub}}^B, v_{z^{lb}}^B), (v_{x^{lb}}^B, v_{z^{ub}}^B) \in E^B_{out}$.

        
        \item Transition Rates: For each $z^{lb} \in Z^B$, ${\cal R}^B({\bf
        p}^B, {\bf x}^B, z^{lb}) = \min\limits_{z \in Z: A(z)=z'} {\cal R}({\bf
        p}, {\bf x}, z)$ (replacing ${\bf p}$ and ${\bf x}$ of the minimal rate
        by the elements in ${\bf p}^B$ and ${\bf x}^B$ respectively, which
        minimize the rate), and ${\cal R}^B({\bf p}^B, {\bf x}^B, z^{ub}) =
        \max\limits_{z \in Z: A(z)=z'} {\cal R}({\bf p}, {\bf x}, z)$ (similarly
        replacing ${\bf p}$ and ${\bf x}$ of the maximal rate by the elements in
        ${\bf p}^B$ and ${\bf x}^B$ respectively, which maximize the rate).
\end{itemize}
    
\end{definition}

\begin{example}
    The bounded abstraction $(\Theta^B, \Omega^B)$ of the stratified SIR model
    defines:
    \begin{eqnarray*}
        V^B_x &=& \{v_{S}^{lb}, v_{S}^{ub}, v_{I}^{lb}, v_{I}^{ub},v_{R}^{lb},
        v_{R}^{ub},\}\\
        V^B_z &=& \{v_{inf}^{lb}, v_{inf}^{ub}, v_{rec}^{lb}, v_{rec}^{ub}\}\\
        E^B_{in} &=& ((v_{inf}^{lb}, v_{S}^{lb}), (v_{inf}^{lb},
        v_{I}^{lb}),(v_{inf}^{lb}, v_{I}^{lb}), (v_{rec}^{lb},
        v_{R}^{lb}),(v_{inf}^{ub}, v_{S}^{ub}), (v_{inf}^{ub},
        v_{I}^{ub}),(v_{inf}^{ub}, v_{I}^{ub}), (v_{rec}^{ub}, v_{R}^{ub})\\
        E^B_{out} &=& ((v_{S}^{lb}, v_{inf}^{ub}),(v_{I}^{lb}, v_{inf}^{ub}),
        (v_{I}^{lb}, v_{rec}^{ub}), (v_{S}^{ub}, v_{inf}^{lb}),(v_{I}^{ub},
        v_{inf}^{lb}), (v_{I}^{ub}, v_{rec}^{lb}))\\
        P^B &=& \{\beta^{lb}, \beta^{ub}, \gamma^{lb}, \gamma^{ub}\}\\
        X^B &=& \{S^{lb},  S^{ub}, I^{lb},I^{ub}, R^{lb},  R^{ub}\}\\
        Z^B &=& \{inf^{lb}, inf^{ub}, rec^{lb}, rec^{ub}\}\\
        {\cal I}^B &=& \left\{ 
            \begin{array}{ll}
                0.9& :S^{lb}\\
                0.9& :S^{ub}\\
                0.1& :I^{lb}\\
                0.1& :I^{ub}\\
                0.0& :R^{lb}\\
                0.0& :R^{ub} \end{array}\right.\\
        {\cal P}^B&=& \left\{ 
            \begin{array}{ll}
                1e{-7}& :\beta^{lb}\\
                2e{-7}& :\beta^{ub}\\
                1e{-5}& :\gamma^{lb}\\
                1e{-5}& :\gamma^{ub}\\
            \end{array}\right.\\
            \\
        {\cal X}^B &=& \left\{ 
            \begin{array}{ll}
                v^{lb}_{x} & : x^{lb} \in X^B\\
                v^{ub}_{x} & : x^{ub} \in X^B \end{array}\right.\\
        {\cal Z}^B &=& \left\{ 
            \begin{array}{ll}
                v_{z}^{lb} & : z^{lb} \in Z^B\\
                v_{z}^{ub} & : z^{ub} \in Z^B \end{array}\right.\\
        {\cal R}^{B} &=& \left\{ 
            \begin{array}{ll}
                \beta^{lb} S^{lb} I^{lb} & : z^{lb}_{inf}\\
                \beta^{ub} S^{ub} I^{ub} & : z^{ub}_{inf}\\
                \gamma^{lb} I^{lb}  & : z^{lb}_{rec}\\
                \gamma^{ub} I^{ub}  & : z^{ub}_{rec} \end{array}\right.\\
    \end{eqnarray*}

    The gradient for the bounded abstraction defines:
    \begin{eqnarray}
        \nabla_{\Theta^B, \Omega^B} = \begin{bmatrix} \frac{dS^{lb}}{dt}\\
                \frac{dS^{ub}}{dt}\\
                \frac{dI^{lb}}{dt}\\
                \frac{dI^{ub}}{dt}\\
                \frac{dR^{lb}}{dt}\\
                \frac{dR^{ub}}{dt} \end{bmatrix} = \begin{bmatrix} -{\cal
            R}^{B}({\bf p}^B, {\bf x}^B, z_{inf}^{ub})\\
            -{\cal R}^{B}({\bf p}^B, {\bf x}^B, z_{inf}^{lb})\\
             {\cal R}^{B}({\bf p}^B, {\bf x}^B, z_{inf}^{lb}) - {\cal
             R}^{B}({\bf p}^B, {\bf x}^B, z_{rec}^{ub})\\
             {\cal R}^{B}({\bf p}^B, {\bf x}^B, z_{inf}^{ub}) - {\cal
             R}^{B}({\bf p}^B, {\bf x}^B, z_{rec}^{lb})\\
             {\cal R}^{B}({\bf p}^B, {\bf x}^B, z_{rec}^{lb})\\
             {\cal R}^{B}({\bf p}^B, {\bf x}^B, z_{rec}^{ub}) \end{bmatrix} =
    \begin{bmatrix} -\beta^{ub} S^{ub} I^{ub}\\
        -\beta^{lb} S^{lb} I^{lb}\\
        \beta^{lb} S^{lb} I^{lb}-\gamma^{ub} I^{ub} \\
        \beta^{ub} S^{ub} I^{ub}-\gamma^{lb} I^{lb} \\
        \gamma^{lb} I^{lb}\\
        \gamma^{ub} I^{ub}
    \end{bmatrix} 
   \end{eqnarray}

\end{example}

% The bounded abstraction defines lower and upper bounds on the abstract state variables.  For example, we derive the upper bound on $\frac{dS}{dt}$ in Equation \ref{eqn:dsdt}:

% \begin{eqnarray*}
%     \frac{dS^{lb}}{dt} &\leq \frac{dS}{dt} &\leq \frac{dS^{ub}}{dt}\\
%     -\beta^{ub} S^{ub} I^{ub} &\leq \frac{dS}{dt} &\leq -\beta^{lb} S^{lb} I^{lb}\\
%     -\max(\beta_1, \beta_2) S^{ub} I^{ub} &\leq \frac{d (S_1+S_2)}{dt} &\leq -\min(\beta_1, \beta_2) S^{lb} I^{lb}\\
%     -\max(\beta_1, \beta_2) S^{ub} I^{ub} \leq -\max(\beta_1, \beta_2) (S_1+S_2) I^{ub} &\leq \frac{d S_1}{dt} +\frac{d S_2}{dt}&\leq -\min(\beta_1, \beta_2) (S_1+S_2) I^{lb}\leq -\min(\beta_1, \beta_2) S^{lb} I^{lb}\\
%     -\max(\beta_1, \beta_2) S^{ub} I^{ub} \leq -\max(\beta_1, \beta_2) (S_1+S_2) I^{ub} &\leq \frac{d S_1}{dt} +\frac{d S_2}{dt}&\leq -\min(\beta_1, \beta_2) (S_1+S_2) I\leq -\min(\beta_1, \beta_2) (S_1+S_2) I^{lb}\leq -\min(\beta_1, \beta_2) S^{lb} I^{lb}
% \end{eqnarray*}

% \begin{eqnarray}
%     \frac{dS}{dt} &=& \frac{d S_1}{dt} +\frac{d S_2}{dt} & Stratify: S\\
%     &=& -\beta_1 S_1 I  -   \beta_2 S_2 I& Stratified Rates\\
%     &\leq& -  \min(\beta_1, \beta_2)S_1 I - \min(\beta_1, \beta_2) S_2 I & Upper bound parameters\\
%     &=& - \min(\beta_1, \beta_2)(S_1 + S_2)I   & Factor: $-I \min(\beta_1, \beta_2) $ \\
%     &=& - \min(\beta_1, \beta_2)S I  & Abstract: ${\cal X}(S_1) = {\cal X}(S_2) = S$ \\
%     &\leq& - \beta^{ub}S^{ub}I^{ub}    & Bound \\
%     &=& \frac{d S^{ub}}{dt}\label{eqn:dsdt}
% \end{eqnarray}

% \begin{eqnarray*}
%     \frac{dI}{dt} &=& I S_1 \beta_1 + I S_2 \beta_2 - I\gamma & Stratified Rates\\
%     &\leq& I S_1 \max(\beta_1, \beta_2) + I S_2 \max(\beta_1, \beta_2) & Upper bound parameters\\
%     &=& I  \max(\beta_1, \beta_2)(S_1 + S_2)  & Factor: $I \max(\beta_1, \beta_2) $ \\
%     &=& I  S \max(\beta_1, \beta_2)  & Abstract: ${\cal X}(S_1) = {\cal X}(S_2) = S$ \\
%     &=& I  S\beta^{ub}  & Bound 
% \end{eqnarray*}

% \begin{eqnarray*}
%     \frac{d S^{lb}}{dt}= - \beta^{ub}S^{ub}I^{ub} \leq \frac{dS}{dt} &\leq& - \beta^{lb}S^{lb}I^{lb} = \frac{d S^{ub}}{dt}\\
%     \frac{d I^{lb}}{dt}=  \beta^{lb}S^{lb}I^{lb} - \gamma^{ub} I^{ub} \leq \frac{dI}{dt} &\leq& \beta^{ub}S^{ub}I^{ub}- \gamma^{lb} I^{lb} = \frac{d I^{ub}}{dt}\\
% \end{eqnarray*}

\section{SIR Example Results}
Results are available in this \href{https://github.com/siftech/funman/blob/sep-monthly-demo/notebooks/abstraction-bounding-demo.ipynb}{notebook}.

\section{SIRHD Example Results}

We measured the time to simulate various formulations of the SIRHD model to highlight the effects of stratification, abstraction, and bounding.  We defined a series of stratification, abstraction, and bounding operations as follows.  Starting with the base model $(\Omega, \Theta)$, we either bound the model $(\Omega^B, \Theta^B)=\text{Bound}(\Omega, \Theta)$, or stratify the model $(\Omega^S, \Theta^S)=\text{Stratify}(\Omega, \Theta)$.  Each stratified model $(\Omega^S, \Theta^S)$ is stratified again differently $(\Omega^{S'}, \Theta^{S'})=\text{Stratify}(\Omega^S, \Theta^S)$, bounded $(\Omega^B, \Theta^B)=\text{Bound}(\Omega^S, \Theta^S)$, or abstracted $(\Omega^A, \Theta^A)=\text{Abstract}(\Omega^A, \Theta^A)$.  Each abstracted model $(\Omega^A, \Theta^A)$ is  bounded $(\Omega^B, \Theta^B)=\text{Bound}((\Omega^A, \Theta^A)$, or abstracted again $(\Omega^{A'}, \Theta^{A'})=\text{Abstract}(\Omega^A, \Theta^A)$.  While it is also possible to stratify an abstracted model, we don't explore this operation here.

Figure \ref{fig:sirhd_experiment_layout} describes how we developed models from the SIRHD base model.  We used several stratifications, and then applied abstractions that reversed the stratifications.  Each abstraction was bounded so that we could simulate the model.  We organized the models in the fashion so that we could demonstrate the time to simulate each model.  In the figure, the models are aligned vertically to indicate which models represent the same level of detail.  For example, the base model and the most-abstract abstraction have the same number of states, transitions, and parameters.  Likewise, reversing the last stratification with the first abstraction results in a model that is the same size as the one prior to the last stratification.  Bounding an abstracted model will increase its size polynomially, whereas stratifying it will increase its size exponentially.  The runtime results tables arrange the models in a similar fashion.  The first column of results increases the model size by stratification from the top to the bottom, and the second column decreases the model size from the top to the bottom.


\begin{figure}
	\includegraphics[width=\linewidth]{fig/sirhd_experiment_layout.pdf}
	\caption{\label{fig:sirhd_experiment_layout} Conceptual relationships between model formulations.  The base model can be stratified a number of times.  The final stratified model is abstracted to reverse the stratifications.  Each abstracted model is bounded so that it can be simulated. The dashed box indicates which model are not simulated.}
\end{figure}

\begin{table}\centering
	\begin{tabular}{|r||r|r||r|r|}
		\hline      & \multicolumn{2}{|c||}{3 Age Levels} & \multicolumn{2}{|c|}{5 Age Levels}                              \\
		\hline	Model & Stratified                          & Bounded                            & Stratified   & Bounded     \\
		            &                                     & Abstracted                         &              & Abstracted  \\	\hline
		0           & 0.57                                & 5.26                               & 0.54         & 5.35        \\
		1           & 0.76                                & 7.18                               & 0.69         & 6.93        \\
		2           & 1.33                                & 5.92                               & 2.86         & 7.43        \\
		3           & 2.18                                & {\bf 9.93}                         & 6.29         & {\bf 22.98} \\
		4           & 3.34                                & 9.50                               & 6.17         & 27.63       \\
		5           & 3.88                                & 14.25                              & 11.46        & 52.81       \\
		6           & 5.36                                & 22.84                              & 19.47        & 108.55      \\
		7           & 7.31                                & 23.12                              & 22.97        & 114.84      \\
		8           & 7.13                                & 28.87                              & 26.41        & 143.46      \\
		9           & 8.14                                & 34.59                              & 39.39        & 200.15      \\
		10          & 9.25                                & 32.94                              & 36.55        & 209.17      \\
		11          & 12.36                               & 41.55                              & 49.47        & 245.51      \\
		12          & 12.96                               & 50.17                              & 58.79        & 319.67      \\
		13          & 11.49                               & 56.79                              & 61.23        & 292.44      \\
		14          & 13.73                               & 67.72                              & 66.92        & 361.31      \\
		15          & {\bf 15.17}                         & -                                  & {}\bf 83.80} & -           \\	\hline
	\end{tabular}
	\caption{\label{tab:sirhd_results}  Runtime in seconds to simulate each model formulation.  Model 0 in the Stratified column is the base model.  Models 1-15 in the Stratified column are successive stratifications.  From model 14 to 0 in the Bounded Abstracted column each is a successive abstraction.  The bolded number in the Stratified column is the fully stratified model that we assume is the starting point for analysis.  The bolded number in the Bounded Abstracted column is the smallest model that allows us to confirm the model constraint is satisfied.}

\end{table}

Table \ref{tab:sirhd_results} lists runtime results in seconds to simulate each SIRHD model variation.  Each instance seeks to check a constraint that assesses whether over 200 days the number of infected is no more than 30\% of the population ($N=1.5e9$).  We use 15 stratifications on the model, three for each of the state variables $S$, $I$, $R$, $H$ and $D$.  The three stratifications involve stratifying each state variable into vaccinated and unvaccinated groups (e.g., $S_{vac}$ and $S_{unvac}$), the vaccinated group into age groups (e.g., $S_{vac,0}$, $S_{vac,1}$, $S_{vac,2}$, ...), and the unvaccinated group into age groups (e.g., $S_{unvac,0}$, $S_{unvac,1}$, $S_{unvac,2}$, ...).  We report results for either three or five age groups.  The model index 0-15 refers to successive stratifications of the base model 0 in the Stratified column.  Similarly, the Bounded Abstracted column refers to successive abstractions of the models, where model 14 is the abstraction of model 15 in the Stratified column, and model 13 is the abstraction of model 14 in the Bounded Abstracted column.

We construct the most abstract model 0 from a series of stratifications and abstractions.  It is an over-approximation of the most stratified model that may allow us to prove the constraint is satisfied.  The abstract bounded model represents lower and upper bounds on the number of infected, and if the upper bound is less than 30\% of $N$ (checked through simulation) then we do not need to simulate the most stratified model 15 (our reference model formulation).  The bounded abstract model 0 can be inconclusive if 30\% of $N$ falls between the lower and upper bounds on the number of infected.  By considering successively less abstract models (i.e., proceeding from model 0 to model 1), we expect the lower and upper bounds to tighten.  The bolded time in the Bounded Abstracted model column is the runtime of the first model formulation (starting from model 0) where we can prove that the upper bound on the number of infected is less than 30\% of $N$.  Therefore, if we need to simulate Bounded Abstracted models 0 through 3, we require 28.29  and 42.69 seconds, respectively, for either three or five age groups.  Compared to the time to simulate the largest Stratified model 15, it takes 15.17 or 82.80 seconds respectively.  The savings due to abstraction reduces runtime by one half when there are five age groups, but doubles the runtime when there are three age groups.  This illustrates the trade-off between simulating several abstract models versus one large stratified model.  When the abstract models can capture the important model dynamics using bounds and the number of collapsed stratification levels is large, then abstraction is a win.  

% 
Base model: set beta based upon population
Bounded Base Model: beta lb/ub are based upon population
Stratified Model: set betas based upon populations
Bounded Stratified Model: bounds are based upon populations
Abstracted Stratified Model: cannot run on own
Bounded Abstracted Stratified Model: set betas based upon populations


\begin{figure}
    \includegraphics[options]{fig/sir_stratified_model.pdf}
\end{figure}

\begin{figure}
    \includegraphics[options]{fig/sir_stratified_sim.png}
\end{figure}

% \section{Stratification Abstraction}
% The SIERHD model from the July monthly demo uses the model summarized by the Petrinet diagram in Figure \ref{fig:seirhd}.

\begin{figure}
    \includegraphics[width=\linewidth]{fig/seirhd}
    \caption{\label{fig:seirhd} SEIRHD Model Petrinet}
\end{figure}

The following transitions connect the variables $S_u$, $S_v$, $E_u$, $E_v$, $I_u$, and $I_v$:

\begin{eqnarray*}
    (I_u, S_u) &\xrightarrow[]{r_1}& (I_u, E_u)\\
    (I_u, S_v) &\xrightarrow[]{r_2}& (I_u, E_v)\\
    (I_v, S_u) &\xrightarrow[]{r_3}& (I_v, E_u)\\
    (I_v, S_v) &\xrightarrow[]{r_4}& (I_v, E_v)\\
    (S_u) &\xrightarrow[]{r_5}& (S_v)\\
    (S_v) &\xrightarrow[]{r_6}& (S_u)\\
    (E_u) &\xrightarrow[]{r_7}& (I_u)\\
    (E_v) &\xrightarrow[]{r_8}& (I_v)
\end{eqnarray*}

Recovering the original, unstratified model corresponds to an abstraction where $S = (S_u, S_v)$, $I = (I_u, I_v)$, and $E = (E_u, E_v)$: 

\begin{eqnarray*}
    (I, S) &\xrightarrow[]{r_1}& (I, E)\\
    (I, S) &\xrightarrow[]{r_2}& (I, E)\\
    (I, S) &\xrightarrow[]{r_3}& (I, E)\\
    (I, S) &\xrightarrow[]{r_4}& (I, E)\\
    (S) &\xrightarrow[]{r_5}& (S)\\
    (S) &\xrightarrow[]{r_6}& (S)\\
    (E) &\xrightarrow[]{r_7}& (I)\\
    (E) &\xrightarrow[]{r_8}& (I)
\end{eqnarray*}

In order for the abstraction to preserve the semantics of the stratified model, it must define $S^t = S_u^t + S_v^t$, $I^t = I_u^t + I_v^t$, and $E^t = E_u^t + E_v^t$ for all time points $t$.  If we look at the definitions for these terms, we have:

\begin{eqnarray*}
    \frac{\partial S_u}{\partial t} &=& - I_u S_u r_1 - I_v S_u r_3 - S_u r_5 + S_v r_6\\
    \frac{\partial S_v}{\partial t} &=& - I_u S_v r_2 - I_v S_v r_4 + S_u r_5 - S_v r_6\\ 
    \frac{\partial S}{\partial t} &=& \frac{\partial S_u}{\partial t}  + \frac{\partial S_v}{\partial t} \\
    &=& - I_u S_u r_1 - I_v S_u r_3 - S_u r_5 + S_v r_6 - I_u S_v r_2 - I_v S_v r_4 + S_u r_5 - S_v r_6\\
    &=& - I_u S_u r_1 - I_v S_u r_3 - I_u S_v r_2 - I_v S_v r_4 \\
\end{eqnarray*}

\begin{eqnarray*}
    \frac{\partial I_u}{\partial t} &=& I_u S_u r_1 - I_u S_u r_1 + I_u S_v r_2 - I_u S_v r_2 + E_u r_7\\
    &=&  E_u r_7\\
    \frac{\partial I_v}{\partial t} &=& I_v S_u r_3 - I_v S_u r_3 + I_v S_v r_4 - I_v S_v r_4 + E_v r_8\\ 
    &=& E_v r_8\\ 
    \frac{\partial I}{\partial t} &=& \frac{\partial I_u}{\partial t}  + \frac{\partial I_v}{\partial t} \\
    &=& E_u r_7 + E_v r_8
\end{eqnarray*}

\begin{eqnarray*}
    \frac{\partial E_u}{\partial t} &=& I_u S_u r_1 + I_v S_u r_3 - E_u r_7\\
    \frac{\partial E_v}{\partial t} &=& I_u S_v r_2 + I_v S_v r_4 - E_v r_8\\ 
    \frac{\partial E}{\partial t} &=& \frac{\partial E_u}{\partial t}  + \frac{\partial E_v}{\partial t} \\
    &=& I_u S_u r_1 + I_v S_u r_3 - E_u r_7 + I_u S_v r_2 + I_v S_v r_4 - E_v r_8
\end{eqnarray*}

Abstraction implies that we allow additional behaviors in the more abstract model (i.e., overapproximate).  In Petrinet models, overapproximation corresponds to cases where the abstract compartment may take on additional values beyond those possible when aggregating the corresponding refined compartments.



\begin{eqnarray*}
    \underline{\frac{\partial S_u}{\partial t}} &\geq&  - \overline{I_u} \overline{S_u} r_1 - \overline{I_v} \overline{S_u} r_3 - \overline{S_u} r_5 + \underline{S_v} r_6\\
    &=&  - N^2 r_1 - N^2 r_3 - N r_5 + 0 r_6\\
    &=&  - S_u^0 (N  r_1 + N r_3 + r_5)\\
    \\
    \overline{\frac{\partial S_u}{\partial t}} &\leq&  - \underline{I_u} \underline{S_u} r_1 - \underline{I_v} \underline{S_u} r_3 - \underline{S_u} r_5 + \overline{S_v} r_6\\
    &=&   - 0 0 r_1 - 0 0 r_3 - 0 r_5 + S_v^0 r_6\\
    &=&  S_v^0 r_6\\
    \\
    \underline{\frac{\partial S_v}{\partial t}} &\geq&  - \overline{I_u} \overline{S_v} r_2 - \overline{I_v} \overline{S_v} r_4 + \underline{S_u} r_5 +-\overline{S_v} r_6\\
    &=&  - N S_v^0 r_1 - N S_v^0 r_4 - 0 r_5 + S_v^0 r_6\\
    &=&  - S_v^0 (N  r_1 + N r_4 + r_6)\\
    \\
    \overline{\frac{\partial S_v}{\partial t}} &\leq&  - \underline{I_u} \underline{S_v} r_2 - \underline{I_v} \underline{S_v} r_4 + \overline{S_u} r_5 -\underline{S_v} r_6\\
    &=&  - 0 0 r_1 -0 0 r_4 + S_u^0 r_5 - 0 r_6\\
    &=&  S_u^0 r_5\\
    \\
    \frac{\partial S}{\partial t} &\leq& \overline{\frac{\partial S_u}{\partial t}}  + \overline{\frac{\partial S_v}{\partial t}}\\
    &\leq& S_v^0 r_6 + S_u^0 r_5\\
    \frac{\partial S}{\partial t} &\geq& \underline{\frac{\partial S_u}{\partial t}}  + \underline{\frac{\partial S_v}{\partial t}}\\
    &\geq& - S_u^0 (N  r_1 + N r_3 + r_5) - S_v^0 (N  r_1 + N r_4 + r_6)
    \\
\end{eqnarray*}

\begin{eqnarray*}
    \underline{\frac{\partial I_u}{\partial t}} &\geq&   \underline{I_u} \underline{S_u} r_1 - \overline{I_u} \overline{S_u} r_1 + \underline{I_u} \underline{S_v} r_2 - \overline{I_u} \overline{S_v} r_2 + \underline{E_u} r_7\\
    &=&  0 0 r_1 - N S_u^0 r_1 + 0 0 r_2 - N S_v^0 r_2 + 0 r_7\\
    &=&  - N (S_u^0 r_1 + S_v^0 r_2)\\
    \\
    \overline{\frac{\partial I_u}{\partial t}} &\leq& \overline{I_u} \overline{S_u} r_1 - \underline{I_u} \underline{S_u} r_1 + \overline{I_u} \overline{S_v} r_2 - \underline{I_u} \underline{S_v} r_2 + \overline{E_u} r_7\\
    &=&  N S_u^0 r_1 - 0 0 r_1 + N S_v^0 r_2 - 0 0 r_2 + N r_7\\
    &=&  N (S_u^0 r_1 + S_v^0 r_2 + r_7)\\
    \\
    \underline{\frac{\partial I_v}{\partial t}} &\geq&  \underline{I_v} \underline{S_u} r_3 - \overline{I_v} \overline{S_u} r_3 + \underline{I_v} \underline{S_v} r_4 - \overline{I_v} \overline{S_v} r_4 + \underline{E_v} r_8\\
    &=&  0 0 r_3 - N S_u^0 r_3 + 0 0  r_4 - N S_v^0 r_4 + 0 r_8\\
    &=&  - N (S_u^0 r_3 + S_v^0 r_4)\\
    \\
    \overline{\frac{\partial I_v}{\partial t}} &\leq&  \overline{I_v} \overline{S_u} r_3 - \underline{I_v} \underline{S_u} r_3 + \overline{I_v} \overline{S_v} r_4 - \underline{I_v} \underline{S_v} r_4 + \overline{E_v} r_8\\
    &=&  N S_u^0 r_3 - 0 0 r_3 + N S_v^0 r_4 - 0 0 r_4 + N r_8\\
    &=&  N (S_u^0 r_3 + S_v^0 r_4 + r_8)\\
    \\
    \frac{\partial I}{\partial t} &\leq& \overline{\frac{\partial I_u}{\partial t}}  + \overline{\frac{\partial I_v}{\partial t}}\\
    &\leq& N (S_u^0 r_1 + S_v^0 r_2 + r_7) + N (S_u^0 r_3 + S_v^0 r_4 + r_8)\\
    &=& N (S_u^0 r_1 + S_v^0 r_2 + r_7 + S_u^0 r_3 + S_v^0 r_4 + r_8)\\
    \frac{\partial I}{\partial t} &\geq& \underline{\frac{\partial I_u}{\partial t}}  + \underline{\frac{\partial I_v}{\partial t}}\\
    &\geq& - N (S_u^0 r_1 + S_v^0 r_2 + S_u^0 r_3 + S_v^0 r_4)
    \\
\end{eqnarray*}

\begin{eqnarray*}
    \underline{\frac{\partial E_u}{\partial t}} &\geq&   \underline{I_u} \underline{S_u} r_1 + \underline{I_v} \underline{S_u} r_3 - \overline{E_u} r_7\\
    &=&  0 0  r_1 + 0 0 r_3 - N r_7\\
    &=&  - N r_7\\
    \\
    \overline{\frac{\partial E_u}{\partial t}} &\leq& \overline{I_u} \overline{S_u} r_1 + \overline{I_v} \overline{S_u} r_3 - \underline{E_u} r_7\\
    &=&  N S_u^0 r_1 + N S_u^0 r_3 - 0 r_7\\
    &=&  N S_u^0 (r_1 + r_3)\\
    \\
    \underline{\frac{\partial E_v}{\partial t}} &\geq&  \underline{I_u} \underline{S_v} r_2 + \underline{I_v} \underline{S_v} r_4 - \overline{E_v} r_8\\
    &=&  0 0 r_2 + 0 0  r_4 -N r_8\\
    &=&  -N r_8\\
    \\
    \overline{\frac{\partial E_v}{\partial t}} &\leq&  \overline{I_u} \overline{S_v} r_2 + \overline{I_v} \overline{S_v} r_4 - \underline{E_v} r_8\\
    &=&  N S_v^0 r_2 +N S_v^0 r_4 - 0 r_8\\
    &=&  N S_v^0 (r_2 + r_4)\\
    \\
    \frac{\partial E}{\partial t} &\leq& \overline{\frac{\partial E_u}{\partial t}}  + \overline{\frac{\partial E_v}{\partial t}}\\
    &\leq& N S_u^0 (r_1 + r_3) +N S_v^0 (r_2 + r_4)\\
    \frac{\partial E}{\partial t} &\geq& \underline{\frac{\partial E_u}{\partial t}}  + \underline{\frac{\partial E_v}{\partial t}}\\
    &\geq& - N (r_7 + r_8)\\
    \\
\end{eqnarray*}

\begin{eqnarray*}
    S^{t+dt} &=& S^t + \frac{\partial S}{\partial t}dt\\
    S^{t+dt} &\leq& S^t + \overline{\frac{\partial S}{\partial t}}dt\\
    &=& S^t + (- \underline{I_u} \underline{S_u} r_1 - \underline{I_v} \underline{S_u} r_3 - \underline{S_u} r_5 + \overline{S_v} r_6)dt
\end{eqnarray*}

Assume that all compartments are population constrained.  Use information about monotonicity.

\[\frac{\partial S_u}{\partial t} \leq 0, 0 \leq S_u \leq N\]
\[\frac{\partial S_v}{\partial t} \leq 0, 0 \leq S_v \leq N\]
\[ 0 \leq I_u \leq N\]
\[ 0 \leq I_v \leq N\]

% \section{Parameter Synthesis as Abstraction Refinement}

% Parameter synthesis involves labeling regions of a parameter space as safe or unsafe.  This can be formulated as an expression:

% \begin{align}
%     \label{eqn:formulation1}\tag{F1}
%      & \exists \posregion \subseteq \reals^n \forall \point \in \posregion. \model(\point) \wedge \query(\point)         & // & \text{true}  \\
%     \label{eqn:formulation2}\tag{F2}
%      & \exists \negregion \subseteq \reals^n \forall \point \in \negregion. \neg \model(\point) \vee \neg \query(\point) & // & \text{false}
%     % \label{eqn:formulation3}
%     %  & \exists \irrelevantregion \subseteq \reals^n \forall \point \in \irrelevantregion. \neg \model(\point)         & // & \text{irrelevant}
% \end{align}
% \noindent where each point $\point$ is an $n$-dimensional vector.  We refer to $\posregion$ and $\negregion$ as the respective ``true'' and ``false'' sets.  We call the pair $(\posregion, \negregion)$ the parameter space of $\model$ and $\query$.



% In addition to the parameters comprising the parameter space, the model includes a set of state variables $V$.  The model defines (via ODE or PDE equations) the state $S$ at various points in time and space.  We augment the state to include the (constant or piece-wise constant) model parameters, as well as structural parameters describing the time horizon and step size.

% An abstraction of the state space groups states into abstract states, and defines abstract transitions between abstract states.  An abstract transition between abstract states indicates that there exists at least one transition between states of the respective abstract states.  

\end{document}