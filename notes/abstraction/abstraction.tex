\begin{definition}
    An abstraction $(\Theta^{A}, \Omega^{A})$ of a Petrinet and the associated
    semantics $(\Theta, \Omega)$ that is produced by the abstraction operator
    $A$ has the following properties:
    \begin{itemize}
        \item State: For each $x \in X$,  there exists an $x' \in
        X^{A}$ where $A(x) = x'$.  
        % For each vertex $v_x \in V_x$,  $A(v_x) = v_x'$ where $v_x' \in
        % V_x'$.   
        % For each $x\in X$ where  ${\cal X}(x) =
        % V_x$, $A(x) = x'$, and $A(v_x) = v_x'$, then ${\cal X}'(x')=
        % v_{x'}'$. 
         For each abstract state variable $x' \in X^{A}$, the initial value is the sum of the initial values of state variables mapped to $x'$ by $A$, so that  ${\cal I}^{A}(x') = \sum\limits_{x \in X: A(x) = x'} {\cal I}(x)$.
        \item Parameters: For each $p \in P$, there exists a $p'\in P^{A}$ where $A(p) = p'$.
        For each abstract parameter $p' \in P^{A}$, the value (or interval) is the sum of all parameters mapped to $p'$ by $A$, so that ${\cal P}^{A}(p') = \sum\limits_{p \in P: A(p) = p'} {\cal P}(p)$.
        \item Transitions: For each $z \in Z$, there exists a $z' \in Z^{A}$ where $A(z) = z'$.
        \item In Edges: For each edge $(z, x) \in E_{in}$, there exists a $(z',
        x')\in E_{in}^{A}$, where $A((z, x)) =
        (z', x')$, $A(x) = x'$, and $A(z) = z'$.
        \item Out Edges: For each edge $(x, z) \in E_{out}$, there exists a $(x',
        z')\in E_{out}^{A}$, where $A((x, z))
        = (x', z')$, $A(x) = x'$, and $A(z) = z'$.

        
        \item Transition Rates: For each $z' \in Z^{A}$, 
        \begin{equation}\label{eqn:agg-flow}
            {\cal R}^{A}({\bf p}', {\bf
        x}', z') = \sum\limits_{z \in Z: A(z)=z'} {\cal R}({\bf p}, {\bf
        x}, z)
    \end{equation}
    \end{itemize}
\end{definition}

\begin{example}\label{ex:abstraction}
    The abstraction $(\Theta^{A}, \Omega^{A})$ of the stratified SIR model defines
    (with the changed elements highlighted by ``*''):
    \begin{eqnarray*}
        A &=& \left\{ 
            \begin{array}{lll}
                S &: S_1 &*\\
                S &: S_2&*\\
                I &: I\\
                R &: R\\
               \beta &: \beta_1&*\\
               \beta &: \beta_2&*\\
               \gamma &: \gamma\\
               inf&: inf_1&*\\
               inf&: inf_2&*\\
               rec&: rec\\
            %    v_S &: v_{S_1}&*\\
            %    v_S &: v_{S_2}&*\\
            %    v_I &: v_{I}\\
            %    v_R &: v_{R}\\
            %    (v_{S}, v_{inf}) &: (v_{S_1}, v_{inf_1})&*\\
            %    (v_{S}, v_{inf}) &: (v_{S_2}, v_{inf_2})&*\\
            %    (v_{I}, v_{inf}) &: (v_{I}, v_{inf_1})&*\\
            %    (v_{I}, v_{inf}) &: (v_{I}, v_{inf_2})&*\\
            %    (v_I, v_{rec}) &: (v_I, v_{rec})\\
            %    (v_{inf}, v_I) &: (v_{inf_1}, v_I)&*\\
            %    (v_{inf}, v_I) &: (v_{inf_2}, v_I)&*\\
            %    (v_{rec}, v_R) &: (v_{rec}, v_R)\\
            \end{array}\right.\\
            {\cal R}^A &=& \left\{ 
            \begin{array}{lll}
                \beta_1 S_1 I +  \beta_2 S_2 I& : z_{inf}&* \\
                \gamma I  & : z_{rec}\\
            \end{array}\right.\\
    \end{eqnarray*}
\end{example}

In Example \ref{ex:abstraction}, the abstraction $(\Theta^{A}, \Omega^{A})$ maps the $S_1$ and $S_2$ state variables to the $S$ state variable (de-stratifying the Petrinet).  In combining the state variables, the abstract Petrinet consolidates the transitions $inf_1$ and $inf_2$ and associated rates from susceptible to infected.  

Like the base model, the abstraction $(\Theta^{A}, \Omega^{A})$ defines a gradient $\nabla_{\Omega^{A}, \Theta^{A}}({\bf p}^{A}, {\bf x}^{A}, t) = (\frac{dx_1'}{dt},
\frac{dx_2'}{dt}, \ldots)^T$, in terms of Equation \ref{eqn:flow}.
Via Equation \ref{eqn:agg-flow}, the abstraction thus expresses the gradient by aggregating terms from the
base Petrinet and semantics.  It preserves the flow on consolidated transitions, but
expresses the transition rates in terms of the base states.  As such, the
abstraction compresses the Petrinet graph structure, but at the cost of
expanding the expressions for transition rates. Moreover, the transition
rates refer to state variables and parameters (e.g., $\beta_1$, $\beta_2$, $S_1$, and $S_2$) that are not expressed
directly by the abstract Petrinet and semantics (e.g., as $\beta$ and $S$), and by extension, the gradient. We address this in the next section.

